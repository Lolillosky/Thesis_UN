%%% Se puede cambiar el nombre de la introduccion (o incluso dejarlo como ``INTRODUCCION''):
\chapter{Introducción}
\graphicspath{{figuras/int/}}
\label{cap:introd}

Normalmente a la introducción se le conoce como ``Estado del arte'' y tiene por objeto resultados anteriores que, generalmente, no ha desarrollado el autor en el presente trabajo.

Esto es la introducción\index{introduccion@introducción} a la relación de Maxwell\index{relacion@relación!de Maxwell} y a la relación de Kramer-Krönig\index{relacion@relación!de Kramer-Kronig@de Kramer-Krönig}. Por otra parte todo esto es muy interesante y el tribunal tendrá que leerselo.

\figura{Figures/int/patata}{6}{Figura en la introducción}

\section{Sección en capítulos numerados y apéndices}

Aquí empieza la sección

\subsection*{Subsección de la introducción}

Aquí empieza la subsección. !`Atención en el asterisco en la definición de la subsección!
