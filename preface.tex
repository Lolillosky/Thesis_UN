%\graphicspath{{figuras/pro/}}

Texto del prólogo, que debería ser pequeño (2 a 8 págs.) indicando cuál es el objeto del trabajo (qué se estudia, en qué se hace enfasis, \dots), por qué es interesante el tema (desde puntos de vista básicos y/o aplicados), qué se ``resuelve'' con este estudio, cuales son las particularidades esenciales y novedades respecto de otros trabajos similares (en enfoque, métodos, punto de vista, \dots), \dots. Finalmente, es interesante en un párrafo indicar someramente la estructura del trabajo (para que uno sepa a qué capítulo debe ir para ver algo, y/o sepa cual es el flujo de conocimientos en el trabajo). Normalmente NO debería haber figuras, salvo que sean de la ``vida cotidiana'' y que ilustren aplicaciones o interés del trabajo (es decir: sólo se pondrían figuras interesantes que no caben en ningún otro lugar).

%\figura{patata}{6}{Figura muy bonita en el prólogo}
