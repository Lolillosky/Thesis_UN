%%% Se puede cambiar el nombre del primer capitulo:
\chapter{Nombre del segundo capítulo}
\graphicspath{{figuras/cap2/}}
\label{cap:sistexp}

Esto es el sistema experimental y ahora fijaos en la figura \ref{fig:patata2}:

%%% primer argumento: label y nombre de fichero sin extension / segundo: ancho en cm / tercero: caption
\figura{Figures/cap2/patata2}{6}{Figura en el capítulo \ref{cap:sistexp}}

Aquí iría más texto \dots pero como no sé que escribir (?) cambio de página \dots

\newpage

\section{Sección}

Pongamos una tabla\index{plantilla}:

\begin{table}[!htp]
\centering
 \begin{tabular}{| r | r || c | c | c |}
	\hline
	1 & 2 & 3 & 4 & 5\\\hline
        a & b & c & d & e\\
        A & B & C & {\bf d} & E\\
	\hline
  \end{tabular}
  \caption{Un pie de tabla normal}\label{tab:primera}
\end{table}

La tabla \ref{tab:primera} es un poco insulsa. Vamos a otra página

\newpage

\subsection*{Subsección}

!`Atención en el asterisco en la definición de la subsección!

\subsection*{Otra subsección}

Esto es parecido a lo del capítulo \ref{cap:introd}, pero aquí veremos una fórmula:

\[
 x^{2}-x=x(x-1)
\]

\noindent\verb|Algunos formatos:|

?`Hace calor? Estamos a 20\grc C. Es muy {\sl importante} la conductividad en el {\it campo} eléctrico~\cite{Eltsov2000}. {\scshape Hola, esto es mi Tesis}.
